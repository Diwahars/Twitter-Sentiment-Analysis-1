%%%%%%%%%%%%%%%%%%%%%%%%%%%%%%%%%%%%%%%%%%%%%%%%%%%%%%%%%
\chapter{ Program source code }
\label{chap:appendix2}

\section{football\_analyzer.py} \label{sec:FootballAnalyzer}
\begin{verbatim}
#!/usr/bin/env python
# encoding: utf-8
"""
twitter_sentiment_analysis.py

Created by Elvar Orn Unnthorsson on 07-12-2011
Copyright (c) 2011 ellioman inc. All rights reserved.
"""

import sys
import os
import twitter
import nltk
import thread
from twitter_aggregator import *
from html_creator import *
from sentiment_analyzer import *

class FootballAnalyzer:

  """
  FootballAnalyzer is searches twitter for tweets, performs a 
  sentiment analysis of each tweet harvested and creates a webpage,
  in the folder \"html\" that shows the results.
  """
  
  def __init__( self, search_terms = [], pages = 3, results_per_page = 50 ):
    """
    Input: search_terms. A list of search terms to for searching Twitter.
    Input: pages. A Number that determines how many pages of tweets to search for.
    Input: results_per_page. A Number which determines how many tweet results 
           should be on each page.
    Constructs a new FootballAnalyzer instance.
    """
    
    self.html_filename = "DataMiningResults.html"
    self.template_filename = "template.html"
    self.task_finished = False
    self.search_terms = search_terms
    self.pages = pages
    self.results_per_page = results_per_page
  
  
  def run( self ):
    """
    run( self ):
    Runs the football analyzer. Searches, analyzes and creates the webpage.
    """
    
    print "=======================\nData mining starting\n=======================\n"
    tweets = self.__search()
    analyzed_tweets = self.__analyze( tweets )
    self.__create_webpage( analyzed_tweets )
    
    print "=======================\nData mining successful\n=======================\n"
  
  
  def __search( self ):
    """
    __search( self ):
    Creates a search aggregator instance and performs a twitter search 
    using the the search parameters given in the constructor. Returns 
    a list of the tweets harvested.
    Return: A list of all tweets harvested.
    """
    
    try:
      print "Searching..."
      self.__start_task();
      self.aggregator = TwitterAggregator()
      self.aggregator.twitter_search( search_terms = self.search_terms, 
                                             pages = self.pages, 
                                  results_per_page = self.results_per_page )
      
      tweets = {} 
      for term in self.search_terms:
        tweets[term] = self.aggregator.get_tweets( search_terms = [ term ], 
                                                   return_all_tweets = True  )
      self.__end_task();
      print "Search complete"
      
      return tweets
    
    except:
      raise Exception ("Unknown error in FootballAnalyzer::__search")






  def __analyze( self, tweets ):
    """
    __analyze( self, tweets ):
    Input: tweets. A list of tweets strings
    Creates a sentiment analyzer instance and uses it to analyze each tweet 
    harvested by the twitter aggregator. it returns a list of the analyzed tweets.
    Return: A list of the tweets analyzed. 
    """
    
    try:
      print "Analyzing the data..."
      self.__start_task();
      self.analyzer = SentimentAnalyzer()
      analyzed_tweets = self.analyzer.analyze( tweets )
      self.__end_task();
      print "Analyzing complete"
      
      self.analyzer.show_most_informative_features( 20 )
      
      return analyzed_tweets
    
    except:
      raise Exception ("Unknown error in FootballAnalyzer::__analyze")
  
  
  def __create_webpage( self, analyzed_tweets ):
    """
    __create_webpage( self, analyzed_tweets ):
    Input: analyzed_tweets. A list of tweets strings
    Creates a webpage with statistics gathered from the tweet aggregator 
    and analyzer, a word cloud with the 30 most used words in the tweets
    and list of each tweet harvested which are colored green, red and white 
    depending on the results from the analyzer.
    """
    
    try:
      print "Creating HTML page..."
      self.__start_task();
      
      # A Statistic dictionary, 
      # used to print out the information on the results webpage.
      stats = {}
      stats["search_parameters"] = self.search_terms
      stats["tweets_count"] = self.aggregator.tweets_count
      stats["positive_count"] = self.analyzer.get_analysis_result( "positive" )
      stats["negative_count"] = self.analyzer.get_analysis_result( "negative" )
      stats["neutral_count"] = self.analyzer.get_analysis_result( "neutral" )
      

      html_page = HTMLCreator( self.html_filename, 
                               self.template_filename, 
                               analyzed_tweets, 
                               stats )
      html_page.create_html()
      self.__end_task();
      
      print "Creating HTML page complete\n"
    
    except:
      raise Exception ("Unknown error in FootballAnalyzer::__create_webpage")
  
  
  def __start_task( self ):
    """
    __start_task( self ):
    Creates a thread which displays dots while a function 
    in the FootballAnalyzer is running.
    """
    
    self.task_finished = False
    thread.start_new_thread( self.__print_time, ( 1.0, ) )
  
  
  def __end_task( self ):
    """
    __end_task( self ):
    Stops the thread created in the __start_task() function.
    """
    
    self.task_finished = True
    time.sleep( 1.0 )
  
  
  def __print_time( self, delay ):
    """
    __print_time( self, delay ):
    Input: delay. A number which determines how much time should be 
                  between the dot printing
    Prints dots while a function in the FootballAnalyzer is running.
    """
    
    while not self.task_finished:
      print "."
      time.sleep( delay )




if __name__ == '__main__':
  search = [ "MUFC LCFC", "Manchester United Liverpool" ]
  page_per_search = 3
  results_on_page = 300
  f = FootballAnalyzer( search_terms = search, 
                               pages = page_per_search, 
                    results_per_page = results_on_page )
  f.run()
\end{verbatim}

\section{twitter\_aggregator.py} \label{sec:TwitterAggregator}
\begin{verbatim}
#!/usr/bin/env python
# encoding: utf-8
"""
twitter_aggregator.py

Created by Elvar Orn Unnthorsson on 07-12-2011
Copyright (c) 2011 ellioman inc. All rights reserved.
"""

import string
import sys
import os
import re
import datetime
import twitter
import time
import redis
from os.path import join as pjoin
import ast


class TwitterAggregator:
  
  """
  TwitterAggregator performs a Twitter GET Search and harvests tweets
  using the search parameters given to it. It saves the twitter data,
  from the search, to a redis database and allows the user to get the
  data by calling the function "get_tweets()".
  """
  
  def __init__( self ):
    """
    Constructs a new TwitterAggregator instance.
    """
    
    self.redis = redis.Redis()
    self.info_to_get = ['text', 'profile_image_url', 'from_user']
    self.search_results = {}
    self.raw_data_directory_name = "raw_mining_data"
    self.filtered_data_directory_name = "filtered_mining_data"
    
    english_file = pjoin( sys.path[0], 
                          "sentiment_word_files", 
                          "Nielsen2010Responsible_english.csv")
    self.analyzeEnglish = dict(map(lambda (w,e): (w, int(e)), \
              [ line.strip().lower().split('\t') for line in open(english_file) ]))
    self.tweets_count = 0
  
  
  def twitter_search( self, search_terms = [], pages = 1, results_per_page = 100 ):
    """
    twitter_search( self, search_terms = [], pages = 1, results_per_page = 100 ):
    Input: search_terms. A list of search terms to for searching Twitter.
    Input: pages. A Number that determines how many pages of tweets to search for.
    Input: results_per_page. Number of tweets results on each page.
    Searches twitter for the things listed in the search_terms list and saves 
    the data collected, in a Redis database.
    """
    
    if search_terms == []: return ''
    
    self.pages = pages
    self.results_per_page = results_per_page
    twitter_search = twitter.Twitter( domain="search.twitter.com" )
    search_results = []
    
    try:
      # For each search term...
      for term in search_terms:
        results = []
        for page in range( 1, pages+1 ):
          results.append(twitter_search.search( q = term, 
                                              rpp = results_per_page, 
                                             page = page, 
                                      result_type = "recent" ) )
        
        # Get the tweets from the search
        new_tweets_ids = self.__get_tweet_ids( search_results=results )
        
        # Save tweets and other information to the database
        term_redis_name = term.replace( " ", "_" )
        term_tweetsIds_name = term_redis_name + "$TweetIds"
        term_searchcount_name = term_redis_name + "$SearchCount"
        


        if self.redis.exists( term_redis_name ):
          current_tweets_ids = ast.literal_eval( 
                                    self.redis.get( term_tweetsIds_name ) )
          current_tweets_ids.append( new_tweets_ids )
          self.redis.set( term_tweetsIds_name, current_tweets_ids )
          self.redis.set( term_searchcount_name, 
                          int( self.redis.get( term_searchcount_name ) ) + 1 )
        else:
          self.redis.set( term_redis_name, True )
          self.redis.set( term_tweetsIds_name, [new_tweets_ids] )
          self.redis.set( term_searchcount_name, 1 )
    
    except:
      raise Exception ("Unknown error in TwitterAggregator::twitter_search")
  
  
  def get_tweets( self, search_terms = [], return_all_tweets = True ):
    """ 
    get_tweets( self, search_terms = [], return_all_tweets = True ):
    Input: search_terms. A list of search terms to fetch from the database.
    Input: return_all_tweets. Boolean to say wether to get all or last tweets.
    Fetches from the database each tweet text, username and url to the user's 
    display pictures for each search term given in the "search_term" parameter.
    Return: A list which contains lists that has each tweet text, 
            username and url to profile picture.
    """
    
    returnList = []
    
    # If the search term list is empty, return an emptylist.
    if search_terms == []: return []
    
    try:
      # If not then get information about each tweet and put it in a list.
      for term in search_terms:
        term_redis_name = term.replace( " ", "_" )
        # Skip if the search term isn't in the database
        if not self.redis.exists( term_redis_name ): 
          print "Error: The search term", term, "has not been searched for before..."
          continue
        
        term_tweetsIds_name = term_redis_name + "$TweetIds"
        tweet_searches = ast.literal_eval( self.redis.get(term_tweetsIds_name) )
        
        if return_all_tweets:
          ids = list( set( [ t_id 
                             for results in tweet_searches 
                             for t_id in results ] ) )
          tweet_info = [ self.redis.get( t_id ) for t_id in ids ]
          
          for t in tweet_info:
            returnList.append( ast.literal_eval( t ) )
            self.tweets_count += 1
        
        else:
          ids = list( set( [ t_id 
                             for t_id in tweet_searches[ len(tweet_searches)-1 ] ] ) )
          tweet_info = [ self.redis.get( t_id ) for t_id in ids ]

          for t in tweet_info:
            returnList.append( ast.literal_eval( t ) )
            self.tweets_count += 1
      
      return returnList
    
    except:
      raise Exception ("Unknown error in TwitterAggregator::__get_tweet_ids")
      return []
  
  
  def __get_tweet_ids( self, search_results = [] ):
    """
    __get_tweet_ids( self, search_term = "", search_results = [] ):
    Input: search_results. A list with the JSON results from the Twitter API
    Fetches the tweet ids from the JSON results.
    Return: A list containing the ids found.
    """
    
    # Return empty list if the list in the parameter is empty
    if search_results == []: return []
    
    count = 0
    tweet_ids = []
    non_english_tweets = 0
    
    try:
      # For each search result...
      for result in search_results:
      
        # For each tweet found...
        for tweet in result['results']:
          # Skip tweets that are not in english
          if not self.__is_english_tweet( tweet['text'] ) :
            continue
            
          tweet_info = []


          # Get each information data that was requested...
          for fetched_data in self.info_to_get:
            if ( type(tweet[fetched_data]) == int): 
              tweet_info.append( tweet[fetched_data] )
            else: 
              tweet_info.append( tweet[fetched_data].encode('ascii', 'ignore') )
            
          # Append the information to the gathered list
          tweet_ids.append( tweet['id_str'] )
          
          # Put the tweet info in the database with the string ID as key
          self.redis.set( tweet['id_str'], tweet_info )
      return tweet_ids
    
    except:
      raise Exception ("Unknown error in TwitterAggregator::__get_tweet_ids")
      return []
  
  
  def __is_english_tweet( self, tweet ):
    """
    __isEnglishTweet( self, tweet ):
    Input: tweet. A string containing a tweet to check.
    Determines whether a comment is an english one or not. This function 
    was given to the author by Helgi who is a fellow student at DTU.
    Return: True if english, False if not not english
    """
    
    try:
      lang = sum(map(lambda w: self.analyzeEnglish.get(w, 0), \
        re.sub(r'[^\w]', ' ', string.lower(tweet)).split()))
        
      if lang >= 1.0: return True
      else: return False
    except:
      raise Exception ("Unknown error in TwitterAggregator::__isEnglishTweet")
      return False









\end{verbatim}

\section{sentiment\_analyzer.py} \label{sec:SentimentAnalyzer}
\begin{verbatim}
#!/usr/bin/env python
# encoding: utf-8
"""
sentiment_analyzer.py

Created by Elvar Orn Unnthorsson on 07-12-2011
Copyright (c) 2011 ellioman inc. All rights reserved.
"""

import sys
import os
from os.path import join as pjoin
import nltk
from nltk.classify import NaiveBayesClassifier
from nltk.corpus import stopwords
import codecs
import re

class SentimentAnalyzer:

  """
  SentimentAnalyzer trains a Naive Bayes classifier so that it can determine
  whether a tweet is positive, negative or neutral. It uses training data that
  was manually categorized by the author. The analyze function should be used
  to classify a list of tweets to analyze.
  """

  def __init__( self ):
    """
    Constructs a new SentimentAnalyzer instance.
    """

    self.results = { "positive": 0, "negative": 0, "neutral": 0}
    self.data = {}
    self.min_word_length = 3

    self.stopSet = set(stopwords.words('english'))
    extra_stopwords = ["he's", "she's", "RT" ]
    for stopword in extra_stopwords: self.stopSet.add( stopword )

    # Naive Bayes initialization
    self.__init_naive_bayes()





  def analyze(self, data):
    """
    analyze(self, data):
    Input: data. A list of tweets to analyze.
    Takes a list of tweets and uses sentiment analysis to determine whether 
    each tweet is positive, negative or neutral.
    Return: List with each tweet categorized with the proper sentiment value.
    """

    return self.__analyse_using_naive_bayes( data )


  def get_analysis_result(self, data_to_get):
    """
    get_analysis_result(self, data_to_get):
    Input: data_to_get. The statistic to get from the results dictionary.
    Gets the count of either positive, negative or neutral from the results 
    dictionary after doing an analysis. 
    Return: The count of positive, negative or positive tweets found.
    """

    return self.results[data_to_get]


  def show_most_informative_features( self, amount ):
    """
    show_most_informative_features( self, amount ):
    Input: amount. How many features should the function display.
    Displays the most informative features in the classifier used
    to classify each tweet.
    """

    self.classifier.show_most_informative_features( amount )


  def __init_naive_bayes( self ):
    """
    __init_naive_bayes( self ):
    Gets the data from the positive, negative and neutral text files.
    Creates and trains the Naive Bayes classifier, using the data, so 
    that it can learn what constitutes a positive, negative or neutral tweet.
    """

    try:
      pos_file = pjoin( sys.path[0], "sentiment_word_files", "tweets_positive.txt")
      f = codecs.open( pos_file, mode="rU", encoding='utf-8')
      positive = [ line.lower().replace("\n" , " ") for line in f ]
      positive = "".join(word[:] for word in positive).split()
      f.close

      neu_file = pjoin( sys.path[0], "sentiment_word_files", "tweets_neutral.txt")
      f = codecs.open( neu_file, mode="rU", encoding='utf-8')
      neutral = [ line.lower().replace("\n" , " ") for line in f ]
      neutral = "".join(word[:] for word in neutral).split()
      f.close

      neg_file = pjoin( sys.path[0], "sentiment_word_files", "tweets_negative.txt")
      f = codecs.open( neg_file, mode="rU", encoding='utf-8')
      negative = [ line.lower().replace("\n" , " ") for line in f ]
      negative = "".join(word[:] for word in negative).split()
      f.close

      posfeats = [( dict( { word.lower() : True } ), 'pos' ) 
                    for word in positive if self.__check_word( word ) ]
      neufeats = [( dict( { word.lower() : True } ), 'neu' ) 
                    for word in neutral if self.__check_word( word ) ]
      negfeats = [( dict( { word.lower() : True } ), 'neg' ) 
                    for word in negative if self.__check_word( word ) ]

      self.classifier = NaiveBayesClassifier.train( posfeats + neufeats + negfeats )

    except:
      raise Exception ("Unknown error in SentimentAnalyzer::__init_naive_bayes")
  
  
  def __check_word( self, word ):
    """
    __check_word( self, word ):
    Input: word. The word to check.
    Looks at a word and determines whether that should be used in the classifier.
    Return: True if the word should be used, False if not.
    """
    if word in self.stopSet \
      or len(word) < self.min_word_length \
      or word[0] == "@" \
      or word[0] == "#" \
      or word[:4] == "http":
        return False
    else:
      return True








  def __analyze_tweet(self, tweet):
    """
    __analyze_tweet(self, tweet):
    Input: tweet. The tweet that should be analyzed.
    Analyses a tweet using the created Naive Bayes classifier.
    Return: The results fromt the classifier. Possible results: 'pos', 'neg' or 'neu'
    """
    try:
      tweet_features = dict([ (word, True) 
              for word in tweet.lower().split() 
              if self.__check_word( word ) ] )
      return self.classifier.classify( tweet_features )
    
    except:
      raise Exception ("Unknown error in SentimentAnalyzer::__analyze_tweet")
      return 'err'


  def __analyse_using_naive_bayes(self, data):
    """
    __analyse_using_naive_bayes(self, data):
    Input: data. A list of tweets to analyze.
    Takes a list of tweets and uses sentiment analysis to determine 
    whether each tweet is positive, negative or neutral.
    Return: A list of the tweets analyzed.
    """
    analyzed_data = {}
    try:
      for search_term, tweet_data in data.iteritems():
        self.results[search_term + "_positive"] = 0
        self.results[search_term + "_negative"] = 0
        self.results[search_term + "_neutral"] = 0
        
        search_term_data = []
        for data in tweet_data:
          temp_data = data
          result = self.__analyze_tweet( data[0] )
          temp_data.append( result )
          search_term_data.append( temp_data )
          
          if (result == 'pos'): self.results[search_term + "_positive"] += 1
          elif (result == 'neg'): self.results[search_term + "_negative"] += 1
          elif (result == 'neu'): self.results[search_term + "_neutral"] += 1
        
        analyzed_data[search_term] = search_term_data
        self.results["positive"] += self.results[search_term + "_positive"]
        self.results["negative"] += self.results[search_term + "_negative"]
        self.results["neutral"] += self.results[search_term + "_neutral"]
      
      return analyzed_data
    
    except:
      raise Exception("Unknown error in SentimentAnalyzer::__analyse_using_naive_bayes")
      return analyzed_data
\end{verbatim}

\section{html\_creator.py} \label{sec:HTMLCreator}
\begin{verbatim}
#!/usr/bin/env python
# encoding: utf-8
"""
html_creator.py

Created by Elvar Orn Unnthorsson on 07-12-2011
Copyright (c) 2011 ellioman inc. All rights reserved.
"""

import sys
import os
import nltk
import random
from cgi import escape
from os.path import join as pjoin
from mako.template import Template


class HTMLCreator(object):

  """
  HTMLCreator creates a HTML webpage that displays statistics, word cloud
  and a list of all tweets harvested. Must provide the class with the following:
  * Name of the html page to create
  * Name of the template for the html to follow. The template must have: 
      * <ul id="cloud"> to place the word cloud
      * <div class="tweets"> to place the tweets
  * Twitter data consisting of list of [tweet text, profile pic, user name] lists
  """

  def __init__(self, page_name, template_name, twitter_data, stats):
    """
    Constructs a new HTMLCreator instance.
    """

    super(HTMLCreator, self).__init__()
    self.page_name = page_name
    self.template_name = template_name
    self.twitter_data = twitter_data
    self.tweets = ""
    self.word_cloud = ""
    self.stats_html = ""
    self.word_cloud_min_frequency = 5
    self.word_cloud_min_font_size = 1
    self.word_cloud_max_font_size = 25
    self.word_cloud_max_words = 30
    self.word_cloud_min_word_length = 3
    self.stats = stats


  def create_html(self):
    """
    create_html(self):
    Creates the webpage used to show the results 
    from the twitter search and analysis.
    """
    try: 
      f = open( pjoin(sys.path[0], "html/template", self.template_name), "r")
      html = f.read()
      f.close

      # Put the stats in the html
      self.__create_stats_info()
      index = html.find('<div id="stats">') + len('<div id="stats">')
      html_before_stats, html_after_stats = html[:index], html[index:]
      html = html_before_stats + self.stats_html + html_after_stats

      # Append the word cloud to the html
      self.__create_word_cloud()
      index = html.find('<ul id="cloud">') + len('<ul id="cloud">')
      html_before_cloud, html_after_cloud = html[:index], html[index:]
      html = html_before_cloud + '\n' + self.word_cloud + html_after_cloud

      # Append the tweets to the html
      self.__create_tweet_list()
      index = html.find( '<div class="tweet-container">')
                         + len('<div class="tweet-container">')
      html_before_tweets, html_after_tweets = html[:index], html[index:]
      html = html_before_tweets + self.tweets + html_after_tweets

      # Create and save the html file
      f = open( pjoin(sys.path[0], "html", self.page_name), "wb")
      f.write(Template("${data}").render(data=html))
      f.close()

    except:
      raise Exception ("Unknown error in HTMLCreator::create_html")


    def __create_stats_info(self):
      """
      __create_stats_info(self):
      Creates the statistics part of the webpage.
      """
      try:
        self.stats_html = '\n' + '\t'*6
        self.stats_html += '<strong>Search parameters</strong>\n'
        self.stats_html += '\t'*6 + '<br/>\n'
        self.stats_html += '\t'*6 + '<ul class="search-parameters">\n'

        for term in self.stats["search_parameters"]:
          self.stats_html += '\t'*7 + '<li>' + term + "</li>\n"

        self.stats_html += '\t'*6 + '</ul>\n'
        self.stats_html += '\t'*6 + 'Tweets count: '
        self.stats_html += str(self.stats["tweets_count"]) + '\n'
        self.stats_html += '\t'*6 + '<br/>\n'
        self.stats_html += '\t'*6 + 'Positive tweets: '
        self.stats_html += str(self.stats["positive_count"]) + '\n'
        self.stats_html += '\t'*6 + '<br/>\n'
        self.stats_html += '\t'*6 + 'Neutral tweets: '
        self.stats_html += str(self.stats["neutral_count"]) + '\n'
        self.stats_html += '\t'*6 + '<br/>'
        self.stats_html += '\t'*6 + 'Negative tweets: '
        self.stats_html += str(self.stats["negative_count"]) + '\n'
        self.stats_html += '\t'*6 + '<br/>'

      except:
        raise Exception ("Unknown error in HTMLCreator::__create_stats_info")


  def __create_tweet_list(self):
    """
    __create_tweet_list(self):
    Creates the tweet listing part of the webpage.
    """

    count = 1
    try:
      # For each result from the search...
      for search_term, results in self.twitter_data.iteritems():
        
        # For each tweet in the results...
        for tweet_data in results:
          
          # Make sure that the comments are in rows of three elements
          if ( count % 3 == 0 ):
            self.tweets += '\n' + '\t'*5 + '<div class="left-tweets">\n'
          else:
            self.tweets += '\n' + '\t'*5 + '<div class="right-tweets">\n'
          
          # See if it's a positive, negative or neutral tweet and 
          # put appropriate html class for the tweet
          if ( tweet_data[3] == "pos" ):
            self.tweets += '\t'*6 + '<div class="tweet-positive">\n'
          elif ( tweet_data[3] == "neg" ):
            self.tweets += '\t'*6 + '<div class="tweet-negative">\n'
          elif ( tweet_data[3] == "neu" ):
            self.tweets += '\t'*6 + '<div class="tweet-neutral">\n'
          
          # Image
          self.tweets += '\t'*7 + '<div class="img">\n' + '\t'*8
          self.tweets +='<img src="' + tweet_data[1] + '" width="50px"/>\n'
          self.tweets += '\t'*7 + '</div>\n'
          
          # Author's name
          self.tweets += '\t'*7 + '<div class="author">\n' + '\t'*8 
          self.tweets += '<a href="http://twitter.com/#!/' + tweet_data[2]
          self.tweets += '">' + tweet_data[2] + '</a>\n' + '\t'*7 + '</div>\n' 
          # Tweet text
          self.tweets += '\t'*7 + '<div class="text">\n' + '\t'*7 \
          self.tweets += tweet_data[0] + '\n' + '\t'*7 + '</div>\n' 
          self.tweets += '\t'*6 + '</div>\n'
          self.tweets += '\t'*5 + '</div>'
          
          if ( count % 3 == 0 ):
            self.tweets += '\n' + '\t'*5 + '<br class="clear">'
          count += 1

      self.tweets += '\n' + '\t'*5 + '<br class="clear">'

    except:
      raise Exception ("Unknown error in HTMLCreator::__create_tweet_list")


  def __create_word_cloud( self ):
    """
    __create_word_cloud( self ):
    Creates the word cloud part of the webpage. Takes the 30 most
    frequent words used and assigns the relevant class to it.
    Some of the code for this function was taken from the
    "the\_tweet\_tweet\_tagcloud\_code.py" from the book 
    "Mining the social web". The code can be found here on github:
     https://github.com/ptwobrussell/Mining-the-Social-Web
    """

    MIN_FREQUENCY = self.word_cloud_min_frequency
    MIN_FONT_SIZE = self.word_cloud_min_font_size
    MAX_FONT_SIZE = self.word_cloud_max_font_size
    MAX_WORDS = self.word_cloud_max_words
    MIN_WORD_LENGTH = self.word_cloud_min_word_length

    try:
      # Get all words from the tweet search and put them in a list
      tweets = []
      for search_term, results in self.twitter_data.iteritems():

        for tweet_data in results:
          tweet_words = tweet_data[0].split()

          # Append the words in lowercase to remove duplicates
          for word in tweet_words:
            tweets.append( word.lower() ) 

      # Compute frequency distribution for the terms
      fdist = nltk.FreqDist([term for t in tweets for term in t.split()])

      # Customize a list of stop words as needed
      stop_words = nltk.corpus.stopwords.words('english')
      stop_words += ['&', '&amp;', '.', '..','...','...', '?', '!']
      stop_words += [':', '"','&quot;', '(', ')', '()', '-', '--']
      stop_words += ["RT"] # Common Twitter words


      # Create output for the WP-Cumulus tag cloud and sort terms by freq
      raw_output = sorted([ [term, '', freq] for (term, freq) in fdist.items()
                  if freq > MIN_FREQUENCY 
                     and term not in stop_words 
                     and len(term) >= MIN_WORD_LENGTH], 
                  key=lambda x: x[2])

      # Scale the font size by the min and max font sizes
      # Implementation adapted from 
      # http://help.com/post/383276-anyone-knows-the-formula-for-font-s
      def weightTermByFreq(f):
        return (f - min_freq) \
               * (MAX_FONT_SIZE - MIN_FONT_SIZE) \
               / (max_freq - min_freq) + MIN_FONT_SIZE

      min_freq = raw_output[0][2]
      max_freq = raw_output[-1][2]
      weighted_output = [[i[0], i[1], weightTermByFreq(i[2])] for i in raw_output]

      # Create the html list <li> for the results page
      myList = []
      for (tag, n, font_size) in weighted_output:
        myList.append( '\t'*7 + '<li class="tag%d">%s</li>' % (font_size, tag) )



      # Minimize the html list to the number specified,
      # randomize it and add it to the word cloud string
      myList = myList[-MAX_WORDS:]
      random.shuffle(myList)
      self.word_cloud = '\n'.join(tag[:] for tag in myList)

    except:
      raise Exception ("Unknown error in HTMLCreator::__create_word_cloud")
\end{verbatim}